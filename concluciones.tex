\chapter{Concluciones y Trabajo Futuro}
\label{champ:concluciones}
\bigskip
\barra
\bigskip
\section{Concluciones}
La programación secuencial tiene muchas ventajas entre ellas, que el desarrollo de software es sencillo y no requiere de un análisis mas allá del de resolver el problema que se esta atacando dejando aun lado las características de las nuevas arquitecturas, obteniendo así una solución con un rendimiento menor en comparación con una solución ya sea concurrente y/o paralela. Al analizar un problema con mayor profundiad nos podría permitir diseñar, desarrollar e implementar soluciones paralelas y/o concurrentes mediante la programación multi-hilo, estas soluciones no solo nos llevan a una ganacia en tiempo sino también en el aprovechamiento de los recursos de las nuevas arquitecturas multi-núcleo. El análisis y diseño de soluciónes paralelas y/o cuncurrentes puede llegar a ser muy complejo debido a que existen problemas donde la solución depende de resultados anteriores, que es donde este tipo de soluciones pueden o no ser adecuadas, esto dependera de las condiciones de competencia que se generen y el manejo de las mismas.\\\\
Los resultados presentados en el Capítulo \ref{champ:resultados}, muestran que es posible mejorar el rendimiento de una solución secuencial de la codificación de audio \textit{WAV} a \textit{MP3}, al procesar la codificación de forma concurente mediante la programación multi-hilo. Este tipo de procesamiento nos da como resultado una mejoria notable en el rendimiento en comparación con la solución secuencial. También se comprobo que el uso de los recursos del sistema y el tiempo de codificación es proporcional al número de hilos utilizados.\\\\
El particionamiento que se hizo el cual se muestra en la Figura \ref{fig:algoritmo} se baso en dividir el problema de tal forma que el proceso de codificación se llevara a cabo de forma concurrente sin generar condiciones de competencia. Dicho particionamiento se baso también en que el software secuencial se compone principalmente de tres partes: lectura de datos, procesamiento y escritura de resultado. La lectura y escritura de datos toman un tiempo proporcinal al del tamaño del archivo a codificar, sin embargo el procesamiento de los datos es el que ocupa la mayor parte del tiempo, dicho tiempo también aumenta de forma proporcinal al tamaño del archivo a codificar. Entonces la parte del procesamiento fue la que se tomo para que se realizara de manera concurrente mediante la programación muti-hilo, teniendo así una solución la cual no solo depende del tamaño de la entrada si no que también de la arquitectura en la cual se esta ejecutando.\\\\
Un diseño paralelo y/o concurrente inadecuado podria penalizar en gran parte el redimiento de la solución, si no es que hasta la obtención de resultados inesperados así como un mal uso de los recursos del sistema.

\section{Trabajo Futuro}
El problema de como implementar soluciones paralelas y/o concurrentes de forma eficiente, es aún un problema abierto. Pero simpre se debe tomar en cuenta que este tipo de solucines simpre existe una parte secuencial y otra paralela. Es posible hacer un análisis mas profundo del problema el cual nos llevaria a encontarar la funcionalidad que consume mas recursos y por lo tanto seria propensa a paralelizarse. Lo anterior no solo nos llevaria a obtener ganancias en tiempo superiores si no tambien llegar a aprovechar al máximo las nuevas arquitecturas. Se podria obtener un redimento mayor combinando el procesamiento multi-núcleo y distribuido. Por ultimo se prodria combinar el procesamiento paralelo con hardware dedicado mediante circuitos reprogramables para obtener una mayor ganacia en tiempo y optimización de recursos.\\\\
Con base a lo anterior, explorar las posibilidades del diseño paralelo y/o concurrente son muchas. Comentaremos algunas. La primera alternativa es agregar soporte para otros tipos de codificaciones y/o hacer posible la modificación de los parametros de codificación, adaptando la solución presentada. Esto volveria al software desarrollado mas robusto en terminos de formatos soportados. La segunda alternativa es realizar un análisis mas a fondo del codificador \textit{MP3} para encontrar las fucionalidades mas costosas y recurrentes de la solución secuencial, y apartir de esta diseñar soluciones que procesen dichas funcionalidades de forma concurrente, esto dependería de la profundidad del análisis realizado. La tercera alternatva sería llevar la solución a una combinación tanto de procesamiento multi-núcleo y distribuido, obtenieno de esta forma una solución aún mas escalable. Por ultimo otra alternativa es realizar una analisis de las funcionalidades mas recurrentes y costosas de la codificación, y a partir de este generar un co-diseño hardware-software, es decir particionar la solución en software dividiendo las funcionalidades mas costosas, y apartir de esto diseñar una implementación en hardware de dichas funcionalidades, este tipo de particionamiento daría como resultado un sistema de colaboración entre el hardware y software en combinación con el procesamiento multi-núcleo presentado en este trabajo, con lo cual se obtendría un rendimiento mayor. Esto es posible por que el diseño, desarrollo e implementación de sistemas hardware-software es ahora más accesible cuando se implanta en un circuito programable que cuente con procesadores programados o incrustados.