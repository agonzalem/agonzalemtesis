\chapter{Conclusiones y Trabajo Futuro}
\label{champ:conclusions}
\bigskip
\barra
\bigskip

En este trabajo se presentaron dos soluciones secuenciales y paralelas para la construcción de redes porosas sujetas al cumplimiento de restricciones geométricas lo cual es una aportación para el estudio más realista de materiales porosos a través de la simulación por computadora. En las soluciones $SA$ y $SH$ se pudo ver que para traslapes pequeños la $SA$ tuvo un mejor rendimiento en términos de tiempo como consecuencia del bajo número de violaciones a las $RG$ y a que la $SH$ realiza una operación de ordenamiento independientemente del traslape utilizado. Sobre las respectivas soluciones paralelas $SPA$ y $SPH$ se obtuvo que esta última siempre tuvo un mejor rendimiento en términos de tiempo respecto a $SH$ y también respecto a $SPA$ independientemente del traslape utilizado en las pruebas.  Por lo tanto nuestros resultados nos indican que la solución paralela $SHP$ es la que dio mejores resultados, por ejemplo utilizando la solución $SPH$ con $64$ hilos y con una configuración $L=100$ y $\Omega=0.110658$ se obtuvo una aceleración en tiempo de hasta $68.48$ veces más rápido que su contraparte secuencial $SH$.\\

Respecto a las redes porosas que se obtuvieron de la $SPH$, estas son las que dieron mejores resultados no solo por la notable ganancia en tiempo sino también por la distribución de los poros a lo largo de la red, sobre esto último cabe destacar que la isotropía de la red mejoraba conforme se utilizaba un mayor número de hilos esto es una consecuencia del propio diseño del algoritmo de la solución $SPH$ en el cual se introdujo una distribución de los datos totalmente dinámica es decir que durante la ejecución los hilos iban cambiando su zona de trabajo y además para los pasos $MMC$-Paralelos cada hilo trabajaba en dos zonas diferentes a la vez.\\

Acerca del trabajo futuro se pude incluir el diseño e implementación de nuestra solución paralela para modelos híbridos tales como clusters/multithreading, multithreading/gpu, clusters/gpu o clusters/multithreading/gpu.




%En este trabajo, una solución básica, BSGR, para simular redes de poros fue propuesta en virtud de la existencia de restricciones geométricas. Este enfoque es una aportación en el estudio de los materiales de poro más realistas. Con el fin de ofrecer una solución eficiente, también propusimos una versión paralela de BSGR. El algoritmo paralelo fue diseñado mediante el uso de un conjunto de hilos que cooperan en la construcción de la red de poros. Una partición de datos dinámica se llevó a cabo entre los hilos de ejecución. Nuestros resultados indican que el algoritmo propuesto en paralelo superó a su homólogo secuencial, por ejemplo, en una configuración con $ L = 100 $, $ \ Omega = 0,13 $ y utilizando $ 16 $ hilos de la aceleración obtenida fue $ 20,6 $ veces más rápido. El trabajo futuro incluye el diseño y la comparación de las diferentes versiones de nuestra solución paralelo usando diferentes arquitecturas como racimos, procesadores de múltiples núcleos y GPU.

%\textbf{Nota: Este apartado esta siendo traducido del artículo que se presentara en el workshop PDSEC 2014 2014 en el Conrgeso IEEE IPDPS 2014, adicionalmente se complementara con resultados de pruebas en progreso}.\\
%
%In this work, a basic solution, BSGR, to simulate pore networks under the existence of Geometrical Restrictions was proposed. This approach is a contribution in the study of more realistic pore materials. In order to offer an efficient solution, we also proposed a parallel version of BSGR. The parallel algorithm was designed by using a set of threads which cooperate in the construction of the pore network. A dynamic data partitioning was performed among the executing threads. Our results indicated that the proposed parallel algorithm outperformed its sequential counterpart, for example in a configuration with $L= 100$, $\Omega=0.13$ and using $16$ threads the acceleration obtained was $20.6$ times faster. Future work includes the design and comparison of different versions of our parallel solution using different architectures like clusters, multi-core processors and GPU's. 