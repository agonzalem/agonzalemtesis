\chapter{Conclusiones y Trabajo Futuro}
\label{champ:conclusions}
\bigskip
\barra
\bigskip

\textbf{Nota: Este apartado esta siendo traducido del artículo que se presentara en el workshop PDSEC 2014 2014 en el Conrgeso IEEE IPDPS 2014, adicionalmente se complementara con resultados de pruebas en progreso}.\\

In this work, a basic solution, BSGR, to simulate pore networks under the existence of Geometrical Restrictions was proposed. This approach is a contribution in the study of more realistic pore materials. In order to offer an efficient solution, we also proposed a parallel version of BSGR. The parallel algorithm was designed by using a set of threads which cooperate in the construction of the pore network. A dynamic data partitioning was performed among the executing threads. Our results indicated that the proposed parallel algorithm outperformed its sequential counterpart, for example in a configuration with $L= 100$, $\Omega=0.13$ and using $16$ threads the acceleration obtained was $20.6$ times faster. Future work includes the design and comparison of different versions of our parallel solution using different architectures like clusters, multi-core processors and GPU's. 