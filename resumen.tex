%\chapter{Resumen}
%\bigskip
%\barra
%\bigskip
%\thispagestyle{empty}
%El estudio de materiales porosos es de gran importancia para un gran número de aplicaciones industriales, con el fin de estudiar características especificas de los materiales o procesos fisicoquímicos de los mismos(ej. simulación in-silico). Este trabajo tiene como base el Modelo Dual de Sitios y Enlaces(DBMS). Bajo este enfoque se tiene que un material poroso se compone por sitios(cavidades, protuberancias) los cuales están interconectados a través de enlaces; cada sitio se interconecta con una serie de enlaces y cada enlace interconecta a dos sitios. En la actualidad, varios algoritmos computacionales se han implementado, sin embargo, solo unos cuantos validan las restricciones geométricas que surgen al conectar un sitio con sus respectivos enlaces en una red porosa. Al validar este tipo de restricciones nos ayudaría a la creación de redes porosas más cercanas a la realidad. En este trabajo, se proponen dos soluciones para la construcción de redes porosas sujetas a restricciones geométricas es decir se validan las restricciones geométricas entre todos los poros de una red porosa, cada algoritmo se implemento de forma secuencial y paralela. Nuestras propuestas paralelas fueron implementadas utilizando OpenMP para que a través de crear un conjunto de hilos (tareas computacionales), estos trabajen de forma simultanea en regiones aleatorias e independientes de una red porosa y de esta forma disminuir sustancialmente los tiempos de creación.


%% OK
\chapter{Resumen}
\bigskip
\barra
\bigskip
\thispagestyle{empty}
El estudio de los medios porosos se realiza con el fin de entender las caracter\'isticas espec\'ificas de los materiales o procesos capilares de los mismos, es de gran importancia para un gran n\'umero de aplicaciones industriales. El Modelo Dual de Sitios y Enlaces (DBMS, por sus siglas en ingl\'es) ha sido una base importante en el desarrollo de simuladores (in-silico) de medios porosos. Bajo este enfoque, se tiene que un material poroso se compone por sitios(cavidades, protuberancias) los cuales est\'an interconectados a trav\'es de enlaces; cada sitio se interconecta con una serie de enlaces y cada enlace interconecta a dos sitios. En la actualidad, varios algoritmos computacionales para la simulaci\'on de redes porosas se han implementado, sin embargo, solo unos cuantos validan el cumplimiento de las restricciones geom\'etricas que surgen al conectar dos enlaces adyacentes de un mismo sitio, en donde no debe existir interferencia espacial entre ellos. El validar este tipo de restricciones nos ayuda a crear redes porosas m\'as realistas; sin embargo, la complejidad algor\'itimica que conlleva el cumplimiento de estas restricciones hace que el tiempo de construcci\'on de una red aumente. En este trabajo, se parte de un algoritmo secuencial desarrollado como parte del Proyecto Interdisciplinario de los Departamentos de Ing. El\'ectrica y de Qu\'imica de la UAM-IZT, el cual genera redes porosas que incluyen las restricciones geom\'etricas; el inconveniente de este algoritmo es que requiere de largo tiempo para construir redes porosas grandes. Para solucionar dicho aspecto, se proponen dos versiones paralelas de construcci\'on de redes porosas, validando las restricciones geom\'etricas entre todos los poros, bas\'andonos en el DBSM. El objetivo de la primer propuesta es paralelizar el algoritmo secuencial anteriormente desarrollado, mientras que la seguna propuesta aplica el M\'etodo de Monte Carlo paralelo en la construcci\'on de una red porosa.  Nuestras propuestas paralelas fueron implementadas bajo un modelo de memoria compartida, usando OpenMP para crear un conjunto de hilos (tareas computacionales) los cuales trabajan de forma simult\'anea en espacios aleatorios e independientes.
