%% OK
\chapter{Resumen}
\bigskip
\barra
\bigskip
\thispagestyle{empty}
El estudio de los medios porosos, con el fin de entender las caracter\'isticas espec\'ificas de los materiales o procesos capilares de los mismos, es de gran importancia para un gran n\'umero de aplicaciones industriales. El Modelo Dual de Sitios y Enlaces (DBMS, por sus siglas en ingl\'es) ha sido una base importante en el desarrollo de simuladores (in-silico) de medios porosos. Bajo este enfoque, se tiene que un material poroso se compone por sitios(cavidades, protuberancias) los cuales est\'an interconectados a trav\'es de enlaces; cada sitio se interconecta con una serie de enlaces y cada enlace interconecta a dos sitios. En la actualidad, varios algoritmos computacionales para la simulaci\'on de redes porosas se han implementado, sin embargo, solo unos cuantos validan el cumplimiento de las restricciones geom\'etricas que surgen al conectar dos enlaces adyacentes de un mismo sitio, en donde no debe existir interferencia espacial entre ellos. El validar este tipo de restricciones nos ayuda a crear redes porosas m\'as realistas; sin embargo, la complejidad algor\'itimica que conlleva el cumplimiento de estas restricciones hace que el tiempo de construcci\'on de una red aumente. En este trabajo, se parte de un algoritmo secuencial desarrollado como parte del Proyecto Interdisciplinario de los Departamentos de Ing. El\'ectrica y de Qu\'imica de la UAM-IZT, el cual genera redes porosas que incluyen las restricciones geom\'etricas; el inconveniente de este algoritmo es que requiere de largo tiempo para construir redes porosas grandes. Para solucionar dicho aspecto, se proponen dos versiones paralelas de construcci\'on de redes porosas, validando las restricciones geom\'etricas entre todos los poros, bas\'andonos en el DBSM. El objetivo de la primer propuesta es paralelizar el algoritmo secuencial anteriormente desarrollado, mientras que la seguna propuesta aplica el M\'etodo de Monte Carlo paralelo en la construcci\'on de una red porosa.  Nuestras propuestas paralelas fueron implementadas bajo un modelo de memoria compartida, usando OpenMP para crear un conjunto de hilos (tareas computacionales) los cuales trabajan de forma simult\'anea en espacios aleatorios e independientes.
